% Draft of final report for ICERM worskhop
% If you want to add a comment with your name attached, insert  
%    \comment{NAME: ... }
% There is also a \todo{...} command.

\documentclass[11pt]{article}

%\usepackage[outerbars]{changebar}
\newenvironment{changebar}{}{}

\usepackage[pagewise]{lineno}
\setlength\linenumbersep{1.0cm}
\linenumbers

\usepackage{color}
\definecolor{darkgreen}{rgb}{0.1,0.5,0.1}
\definecolor{darkblue}{rgb}{0.1,0.1,0.7}
\usepackage[colorlinks=true,linkcolor=darkblue,citecolor=darkblue,
            filecolor=darkblue,urlcolor=darkblue]{hyperref}

\newcommand{\ignore}[1]{}  % use to comment out blocks of text

\usepackage{ifthen}
% set boolean to false to suppress printing TODOs and comments.
\ifthenelse{\boolean{true}}{%
    \newcommand{\todo}[1]{{\color{red} [TODO: #1]}}
    \newcommand{\comment}[1]{{\color{blue} [#1]}}
   }{%
    \newcommand{\todo}[1]{}
    \newcommand{\comment}[1]{}
   }


% Set helvetica as the default font family:
\RequirePackage{helvet}
\renewcommand\familydefault{phv}

\setlength{\textwidth}{6.0in}
\setlength{\oddsidemargin}{0.3in}
\setlength{\evensidemargin}{0in}
\setlength{\textheight}{8.9in}
\setlength{\voffset}{-1.0in}
\setlength{\headsep}{26pt}
\setlength{\parskip}{5pt}


\begin{document}
\begin{centering}
{\large\bf Report on the ICERM Workshop
\vskip 5pt
\Large
Reproducibility in 
\vskip 3pt
Computational and Experimental Mathematics}
\vskip 5pt
{\large December 10--14, 2012}
\vskip 10pt
Written collaboratively by workshop participants\footnote{%
For a list of participants see the workshop webpage
\url{http://icerm.brown.edu/tw12-5-rcem}.  \\
All content is released under the 
\href{http://creativecommons.org/licenses/by/3.0/}{Creative Commons CC BY
3.0 license}. \\
Version of \today.}
\vskip 5pt
\end{centering}
\vskip 10pt


{\bf Abstract.} 
Science is built upon foundations of theory and experiment engaged through
open, transparent communication. If 
done correctly, scientific theory is testable in a reproducible manner.
The ``reproducible research'' movement in computational science and
mathematics encourages this community and all scientists who use
computational tools to benefit from 
this age-old, proven methodology by adopting better work habits and
publication standards.  

This report summarizes discussions that took place during the 
ICERM Workshop on Reproducibility in Computational and Experimental
Mathematics, held December 10-14, 2012.
The themes and conclusions of the workshop can be briefly summarized as:
\begin{enumerate} 

\item To maximize future scientific progress, 
it is important to promote a culture change that will integrate 
computational reproducibility and a greater degree of openness
into the research and publication process.

\item Journals, funding agencies, and employers 
should support this culture change.

\item A diverse set of tools are available to aid in this enterprise.

\item Reproducible research practices and the use of appropriate tools
should be taught as standard operating procedure in relation to
computational aspects of research.
\end{enumerate}

\vskip 10pt

\section{Introduction} \label{sec:intro}


The emergence of powerful new computational hardware, combined with
a vast array of computational software, presents unprecedented
opportunities for researchers in mathematics and science.  Computing
is no longer merely the “third leg” in the stool of modern science.
In addition to the potential for producing new results in its own
right, it is often the backbone of both theory and experiment, and
a critical component in data analysis and interpretation.

Unfortunately the scientific culture surrounding computational work
has evolved in ways that often make it difficult to efficiently
build on past research, or even to apply the basic tenets of the
scientific method to computational procedures.  Laboratory scientists
are taught to keep careful lab notebooks documenting all aspects
of the materials and methods they use and their negative as well
as positive results, but computational work is often done in a much
less careful, transparent, or well-documented manner. Often there
is no record of the workflow process or the code actually used to
obtain the published results, let alone a record of the false starts.
This ultimately has a negative effect on researchers' own productivity,
their ability to build on past results or participate in community
efforts, and their credibility with other scientists and the public.

There is increasing concern with the current state of affairs in
computational science and mathematics, and growing interest in the
idea that doing things differently can have a host of positive
benefits that will more than make up for the effort required to
learn new work habits.  This research paradigm is often summarized
in the computational community by the phrase ``reproducible research".
Of course some researchers have been doing this for years, but
recent interest and improvements in computational power have led
to a host of new tools developed to assist in this process.  At the
same time there is growing recognition among funding agencies,
policy makers, and the editorial boards of scientific journals 
of the need to support and encourage this movement.
A number of workshops have recently been held on related topics, including a
Roundtable at Yale Law School \cite{??} 
a workshop as part of the Applied Mathematics Perspectives 2011 conference
\cite{??}, 
and several minisymposia at other
conferences, including SIAM Conferences on Computational Science and
Engineering and ICIAM 2011.

The ICERM Workshop on Reproducibility in Computational and Experimental
Mathematics, held December 10-14, 2012, provided an opportunity for
a broad cross section of mathematicians and computational scientists
to discuss many of these issues and possible ways to improve on
current practices.  The first two days of the workshop focused on
introducing the themes of the meeting and discussing policy and
cultural issues.  In addition to introductory talks and open
discussion periods, there were panel discussions on funding agency
policies and on journal and publication policies.  The final three
days featured many talks on tools that help achieve reproducibility
and other more technical topics in the mornings. Afternoons were
devoted to breakout groups discussing specific topics in more depth,
which resulted in several sets of recommendations and other outcomes.
Breakout group topics included: reproducibility tools, funding
policies, publication policies, numerical reproducibility, taxonomy
of terms, reward structure and cultural issues, and teaching
reproducible research techniques.

This document reports on some of the discussions and recommendations
from the workshop.
A set of appendices gives more details on some topics, and numerous
references are collected on the wiki, which can be reached from the workshop
webpage, \url{http://icerm.brown.edu/tw12-5-rcem}.  This webpage and the
wiki also contain slides from talks and breakout group reports.
\todo{include snapshot of wiki as an appendix?}

The terms ``reproducible research" and ``reproducibility" are used in many
different ways to encompass diverse aspects of the desire to make research
based on computation more credible and extensible.  Lively discussion over
the course of the workshop led to some suggestions for terminology, as
discussed further in Appendix~\ref{sec:taxonomy}.
We encourage others who use such
terms in their work to clarify what they mean in order to avoid confusion. 

\todo{How much discussion of taxonomy to put here vs. in appendix?}

Not all aspects of reproducibility could be discussed in the workshop,
and much of the focus was on tools to aid in replicating past
computational results (by the same researcher and/or by others) and
for assisting in tracking the provenance of results and the workflow
used to produce figures or tables, along with discussion of 
policy issues in this connection.  

There was also discussion of
the related but somewhat distinct topic of ``numerical reproducibility''
of computational results when the same program may give different results
due to hardware or compiler issues, particular in the context of parallel 
computing.  This is discussed in Appendix~\ref{sec:numerical}.

\section{Changing the culture and reward structure} \label{sec:rewards}

Workshop participants agreed that cultural changes need to take
place within the field of computationally based research if
reproducibility is to be fostered and maintained.  We should work
towards a culture where the default mode of computational science
is open and transparent, to encourage collaboration in the most
effective manner possible.

Researchers need to be persuaded that their efforts to ensure
reproducibility will reward them with increased productivity --- less time
wasted in trying to recover data that was lost or misplaced, less time
wasted trying to double-check results in the manuscript with data in output
files, and less time wasted trying to determine whether other published
results (or even their own) are truly reliable.  Open access to both primary
and auxiliary source code provides the basis for research to be conducted
transparently with the opportunity to build upon previous work, in the same
manner as open software provided the basis for Linux.  This practice enables
researchers both to benefit fully from the creative energies of the global
community and to participate fully in it.  Most great science is built upon
the discoveries of preceding generations and open computational science
allows this tradition to follow this well-worn path of greatness.  
Researchers should be encouraged to recognize the potential benefits of
openness and reproducibility.

It is also important to recognize that in the short term at least there are
costs and barriers to working in this manner, particularly when the culture
does not recognize the value of developing this new paradigm or the effort
that can be required to develop or learn to use suitable tools.
This is of particular concern to young people who are concerned about
earning tenure or securing a permanent position.   To encourage more movement
towards openness and reproducibility, it is crucial that such work be
acknowledged and rewarded, or at least not penalized.  The current system,
which places a great deal of emphasis on the number of journal publications 
and virtually none on reproducibility (and often too little on 
related computational issues such as verification and validation), penalizes
authors who spend extra time on a publication rather than doing the minimum
required to meet current community standards.  

Changes in the culture could be accelerated by finding ways to give public
recognition to good work in this direction.  
One suggestion is that journals and/or professional societies institute a
yearly award, to be awarded to investigators for excellent reproducible
practice.  Such awards are highly motivating to young researchers in
particular, and potentially could result in a sea change in attitudes. This
award could also be a cross-conference and journal award; the collected list
of award recipients would both increase the visibility of researchers
following good practices and provide examples for others.

The following two sections and the appendices give some additional 
ideas for ways in which funding agencies and journals could help encourage
and recognize reproducibility.  While too many requirement imposed from
above could have a negative effect on the community, we believe they can
take positive steps to reward desirable behavior and help to change the
culture and expectations.

\comment{LeVeque:  Does the following paragraph belong here? Maybe better to
leave out of main report and find a place in the appendices?}
More generally, it is unfortunate that software development is often
discounted in the scientific community, and programming is treated
as something to spend as little time on as possible.  Serious
scientists are not expected to spend time carefully testing code,
let alone documenting it, in the same way they are trained to
properly use other tools or document their experiments.  It has
been said in some quarters that writing a large piece of software
is akin to building infrastructure such as a telescope rather than
to doing science with it, and not worthy of tenure or comparable
status at a research laboratory.  This attitude must be changed in
the community if we are to encourage young researchers to specialize
in computing skills that are essential for the future of mathematical
and scientific research.  We believe the more proper analog to a
large scale scientific instrument is a supercomputer,  whereas
software reflects the intellectual engine that makes the supercomputers
useful, and has scientific value beyond the hardware itself.
Important computational results, accompanied by verification,
validation, and reproducibilty, should be accorded with honors
similar to a strong publication record [Meyer2009; Patterson1999].


\section{Standards of publication and journal policies} \label{sec:pubs}
We feel there is a need to produce a set of ``best practices'' 
for publication of computational results (or any scientific results in which
computation plays a role, for example in data processing, statistical
analysis, or image manipulation).   
These may need to be tailored to different communities, but one
central concern, for which there was almost unanimous agreement by
the workshop participants, is the need for full disclosure of salient
details regarding software use. This should include details of the
algorithms employed (or references), the hardware and software environment,
the testing performed, etc., and would ideally include availability of the
relevant computer code and data with a reasonable level of documentation and
instructions for repeaing the computations performed to obtain the results
in the paper. Appendix~\ref{sec:pubs2} contains a more complete list of
suggestions.

\subsection{Recognizing constraints and goals}
It is recognized that including such details in submitted manuscripts (or,
at the least, in supplementary materials hosted by the journal) 
will be a significant departure
from established practice, where few such details are typically presented.
But science is also changing and becoming more complex and these changes
will be required if the integrity of the computational literature is to be
maintained.
Computational approaches have become central to
science and cannot be completely documented and transparent without the
full disclosure of computational details.  

At the same time, workshop participants agreed that some flexibility must be
exercised.  Very rigorous verification and validity testing, along with a
full disclosure of computational details, should be required of papers making
important assertions, such as the computer-assisted proof of a long-standing
mathematical result, new scientific breakthroughs, or studies that will be
the basis for critical policy decisions.  
On the other hand, a
somewhat more informal approach might be satisfactory when the results are
not of such gravity, or where confidentiality of proprietary, medical,
security or personal information must be maintained.  In other words,
exceptions to the above standards may be appropriate in certain cases, so
long as such exceptions are clearly disclosed by the authors and agreed to
by reviewers and editors.

Some related issues in this arena include: (a) anonymous versus public
review, (b) persistence (longevity) of code and data that is made publicly
available, and (c) how code and data can be ``watermarked'', so that instances
of unauthorized usage (plagiarism) can be detected and (d) how to adjudicate
disagreements that inevitably will arise.

Proper consideration of openness constraints can enable a greater community
to participate in the goals of reproducible research. This includes
copyright, patent, medical privacy, personal privacy, security, and export
issues.

The copyright issue is pervasive in software and data, but many methods and
categories have been established for software licensing with some
granularity such as open licenses, public domain dedication, ...
[Stodden09]. The current explosion in patent conflicts requires the same
developments in handling patent rights, including the assertion of prior art
against predatory patents by public disclosure [cf.
http://www.researchdisclosure.com/]. There is also the need for a fully
commercial policy with avenues to allow audit such as non-disclosure
agreement (NDA) and independent agent for auditing similar to financial audits.
\comment{LeVeque: Do we need this paragraph here or move to an appendix with
more about copyright and licenses?}

Limits to disclosure of data also include issues such as release of
individual data for medical records (HIPAA: [HIPAAref]), census data and
even Google search data that
limit data release except in the aggregate. Of course ``the aggregate'' is
defined differently in each domain.
We also recognize that cultural and legal standards in different
jurisdictions (e.g. European Union, United States, Japan) are significantly
different and that each individual needs to apprise themselves of the most
substantial differences [IMU2010, Hodges2011].

In terms of reproducibility, openness is not the goal but the means. It is
sometimes easiest if the entire process is open, but the quantity of
material can be an obstacle in itself. Thus it may be better to identify the key
parts of the research and fully disclose the parts critical to the
scientific process. 

\subsection{Responsibilities of reviewers and journals}

Workshop participants agreed that changes are needed in policies and
procedures followed by journals, editors and reviewers.
To begin with, it is important that a set of standards for reviewing papers
in the computational arena be established.  Such a set of standards might
include many or all of the items from a ``best practices'' list, 
together with a rational
procedure for permitting exceptions or exclusions.  Additionally, provisions
are needed for referees to obtain access to auxiliary information such as
computer codes and for obtaining assistance, if required, in running
computational tests of the results in submitted papers.

Along these lines, it may be judged permissible for  the computational
claims of a  manuscript
to be verifiable at another site, suggested by the
authors, or on another computer system with a similar configuration.  Such
provisions and regulations are among the many issues in this arena that
deserve discussion in the community.

As emphasized in the previous section, some flexibility is in order. 
Different journals may well adopt somewhat different review standards.
But it is important that certain minimal standards of reproducibility and
rigor be maintained in all refereed journal publications.

\comment{LeVeque: Perhaps mention movement towards a working group
representing several mathematical sciences professional societies
that might propose a set of best practices?}

There is also a need for better standards on how to 
include citations for software and data in the
references of a paper, instead of inline or as footnotes.
Proper citation is important both for improving reproducibility and in order
to provide credit for work done developing software and producing data,
which is a key component in encouraging the desired culture change.


\section{Funding agency policies} \label{sec:funding}

Workshop participants suggested that funding sources, both government
agencies and private foundations, consider establishing some reasonable
standards for proposals in the arena of mathematical and scientific
computing.  If such standards can be common among related agencies, or at
least agree on some basic details, this would significantly simplify the
tasks involved in both preparing proposals and reviewing proposals.

For example, workshop participants recommend that software and data be ``open
by default" unless it conflicts with other considerations. 
Proposals involving computational work might
be required to provide details such as:

\begin{itemize} 
\item Extent of computational work to be performed.
\item Platforms and software to be utilized.
\item Reasonable standards for software documentation and reuse (some agencies
already have such requirements 
\url{http://www.jisc.ac.uk/media/documents/programmes/preservation/spsoftware_report_redacted.pdf}.
\item Reasonable standards for persistence of resulting software.
\item Reasonable standards for sharing resulting software among reviewers and
other researchers.
\end{itemize} 


In addition, we suggest that funding
agencies might add ``reproducible research'' to the list of specific examples
that proposals could include in their Broader Impact statement. There were
also suggestions that templates for data management plans could be made
available that include making software open and available.
\comment{LeVeque: I think such things exist already... I'll look through
what's available at
\url{http://guides.lib.washington.edu/content.php?pid=259952&sid=2660743}}

Other suggestions from breakout group: 
\begin{itemize} 
\item    Set default to open – encourage statements from societies
and others;
\item    Fund training workshops on reproducibility;
\item    Fund cyberinfrastructure for reproducibility at scale, for
both large projects and many long-tail research efforts.
\item    Give publicity and stories to provide clarity on RR concept.
\end{itemize} 


\todo{ Mention tools that specifically help with issues of preservation and
archiving of software/data – tools that aid funding agencies goals as we’ve
outlined them.}

\section{The teaching and training of reproducibility skills}
\label{sec:teaching}

Proficiency in the skills required to carry out
reproducible research in the computational sciences should be taught as part
of the scientific methodology, along with teaching modern programming and
software engineering techniques. This should be a standard part of any
curriculum, just as experimental or observational scientists are taught
to keep a laboratory notebook and follow the scientific method.  
Adopting appropiate tools (see Appendix~\ref{sec:tools}) should be
encouraged, if not formally taught, during the training and mentoring of
students and postdoctoral fellows. Without a change in culture and
expectations at this stage, reproducibility will likely never enter the
mainstream of mathematical and scientific computing.

We see at least five separate ways in which these skills can be
taught: full academic courses, incorporation into existing
courses, workshops and summer schools, online self-study materials, and last
but certainly not least, teaching-by-example on the part of mentors. 

Although a few full-scale courses on reproducibility have been attempted
(see the wiki for links), we
recognize that adding a new course to the curriculum or the students'
schedules is generally not feasible.  Moreover a course on reproducibility
in isolation may not make sense.  It seems more effective as well as more
feasible to incorporate teaching the tools and culture of reproducibility
into existing courses on various subjects, concentrating on the tools most
appropriate for the domain of application.  For example, several workshop
participants have taught classes in which version control is briefly
introduced and then students are required to submit homework by pushing to a
version control repository as a means of encouraging this habit.

A list of potential curriculum topics on reproducibility are listed in 
Appendix~\ref{sec:teaching2}.
Ideally, courseware produced at one institution should be shared with others
under an appropriate public license (e.g. CC BY) and upgraded if multiple
institutions adopt it. 

\section{Conclusions} \label{sec:conclusions}

Computational reproducibility attempts to provide a rock solid foundation to
computational science, much like a rigorous proof is the foundation of
mathematics.  The best reason for going down this path is its connection to
the scientific method in its numerous forms.  Science is built upon
foundations of theory and experiment engaged through open, transparent
communication and, if done correctly, is testable in a
reproducible manner.  Computationally-enabled
science could benefit more from this age-old proven methodology.  
The recommendations in this report and the appendices are intended to help
move the community in this direction.

\todo{This could be improved.}


\bibliographystyle{plain}
\bibliography{references}

\clearpage
\appendix
\centerline{\Large\bf Appendices}
\vskip 10pt

These appendices contain some additional material developed during the
workshop.  Many links were also collected on the wiki, which should be
referred to for more examples of the things discussed here, additional
tools, articles, editorials, etc.  The wiki can be reached from a link at
the bottom of the main workshop webpage, 
\url{http://icerm.brown.edu/tw12-5-rcem}.

\section{The importance of reproducibility} \label{sec:importance}

Recent exponential increases in computing
capabilities present unprecedented challenges to certify
that the results of computations are sound.  There is little point
in setting new records for performance if the results are suspect
because of flaws in design or execution.  Incorrect computations
can potentially impede scientific progress rather than enhancing
it, lead to bad policy, safety, or medical decisions with serious
consequences, and provoke distrust and confusion among other
scientists and the public.


Casadevall and Fang, two prominent biologists, recently wrote in \cite{??}: 
``Although
scientists have always comforted themselves with the thought that science is
self-correcting, the immediacy and rapidity with which knowledge
disseminates today means that incorrect information can have a profound
impact before any corrective process can take place.'' Steen (2011) analyzed
the cause of retraction for 788 retracted papers and found that error was
responsible for 545 (69\%) cases. It seems likely these were predominantly
computational and statistical errors. [Yu2013]


\todo{The scientific method in the computational age?}

\section{Terminology and taxonomy} \label{sec:taxonomy}

The terms ``reproducible research" and ``reproducibility" are used in many
different ways to encompass diverse aspects of the desire to make research
based on computation more credible and extensible.  Lively discussion over
the course of the workshop has led to some suggestions for terminology,
listed below.  We encourage authors who use such
terms in their work to clarify what they mean in order to avoid confusion. 

\todo{Decide what terms to list here...}

There are several possible levels of reproducibility, and it seems valuable
to distinguish between the following:

\begin{itemize} 
\item {\em Reviewed Research.} The descriptions of the research methods have been
independently assessed and the results judged credible. (This includes both
traditional peer review and community review, and does not necessarily imply
reproducibility.)

\item {\em Replicable Research.}  Tools are made available that would allow one to
duplicate the results of the research, for example by running the authors'
code to produce the plots shown in the publication. (Here tools might be
limited in scope, e.g., only essential data or executables, and might only
be made available to referees or only upon request.)

\item {\em Confirmable Research.} The main conclusions of the research can be attained
independently without the use of software provided by the author. (But using
the complete description of algorithms and methodology provided in the
publication and any supplementary materials.)

\item {\em Auditable Research.}  Sufficient records (including data and software) have
been archived so that the research can be defended later if necessary.  The
archive might be private, as with traditional laboratory notebooks.

\item {\em Open Research.}  Well-documented and fully open tools are publicly available
(e.g., all data and open source software) that would allow one to (a) fully
audit the computational procedure, (b) replicate and also independently
reproduce the results of the research, and 
(c) extend the results or apply the method to new problems.

\end{itemize} 


Other terms that often arise in discussing reproducibility have specific
meanings in computational science.  In particular the widely-used acronym
V\&V makes it difficult to use "verify" or "validate" more generally.  These
terms are often defined as follows: 

\begin{itemize} 
\item {\em Verification.}  Checking that the computer code correctly solves the
mathematical problem it claims to solve. (Does it solve the equation right?)

\item {\em Validation.}  Checking that the results of a computer simulation agree with
experiments or observations of the phenomenon being studied.  (Does it solve
the right equation?)
\end{itemize} 

The term "Uncertainty Quantification (UQ)" is also commonly used in
computational science to refer to various approaches to assessing the
effects of all of then uncertainties in data, models, and methods on the
final result, which is often then viewed as a probability distribution on a
space of possible results rather than a single answer.  This is an important
aspect of reproducibility in situations where exact duplication of results
cannot be expected for various reasons.

The {\em provenance} of a computational result is a term borrowed from the
art world, and refers to a complete record of the source of any raw data
used, the computer programs or software packages employed, etc.
\todo{Improve this description!}

\section{Best practices for publications} \label{sec:pubs2}

\todo{Need to clean up this section.}

A number of suggestions were made regarding best practices for publications.
To aid in reproducibility they should contain:

\begin{itemize} 
\item  A precise statement of assertions to be made in the paper.
\item  A statement of the computational approach, and why it constitutes a
rigorous test of the hypothesized assertions.
\item  Complete statements of, or references to, every algorithm employed.
\item  Salient details of auxiliary software (both research and commercial
software) used in the computation.
\item  Salient details of the test environment, including hardware, system
software and the number of processors utilized.
\item  Salient details of data reduction and statistical analysis methods.
\item  Adequacy of parameters such as precision level and grid resolution.
\item  Full statement (or at least a valid summary) of experimental results.
\item  Verification and validation tests performed by the author(s).
\item Availability of computer code, input data and output data, with some
reasonable level of documentation.
\item Instructions for repeating computational experiments described in the
paper.
\item Licensing issues. Ideally all these items ``default to open'',
 i.e. a permissive re-use license, if nothing opposes it.
\item Avenues of exploration examined throughout development, including
information about negative findings.
\end{itemize} 

Referreeing could perhaps be ``modernized" through the use of technology to
allow faster communication like IM or direct email while
maintaining the demands of anonymity.  This would allow simple questions and
answers and perhaps even real-time revision of articles.  The point would be
to increase quality and reduce time to publish, or at least unnecessary time
lag associated with the current system. \comment{Is this sufficiently
relevant to reproducibility?}

Along this line, journals might follow the lead of certain forums, such as
SIGMOD [Bonnet2011], Biostatistics, IPOL and others, that have adopted some
requirements for reproducibility. 

Journal recommendations from breakout group:

\begin{itemize} 
\item Create reproducible overlay issues for journals
\item awards for reproducible papers / certification
\end{itemize} 

Mention IPOL / RunMyCode and other tools specifically aimed at journal
publication.

Further suggested guidelines from breakout groups:
\todo{Need to clean up this list, elaborate?}

\begin{itemize} 
\item    put your name on everything, including code/data;
\item    bibliography includes all code/data used, even your own;
\item    recommend awards for software (and data) contributions;
\item    include software explicitly in grant proposals.

\item   include peer review to increase quality.
\item   Provenance/version control, issues of changing the data after its
creation without a trail.
\item   Auditable, Reversible, data protection, attribution
\item   Persistence (data longevity)
\item   What is the difference between access and openness (do you have access
to Matlab or a supercomputer?)
\item   Up/down votes by peers, curation
\item   Anonymous versus public review
\item   What are the requirements/attributes of shared Data, Source code
\end{itemize} 



\section{Numerical reproducibility} \label{sec:numerical}
One of the foundations of reproducibility is how to deal with (and set
standards for) difficulties such as numerical round-off error and numerical
differences when a code is run on different systems or different numbers of
processors.  Such difficulties are magnified as problems are scaled up to
run on very large, highly parallel systems.  Increasingly, it is difficult
to determine whether a code has been correctly ported to a new system,
because computational results differ from standard benchmark cases.

Computations on a parallel computer system present particularly acute
difficulties for reproducibility since, in typical parallel usage, the
number of processors may vary from run to run.  Even if the same number of
processors is used, computations may be split differently between them or
combined in a different order.  Since computer arithmetic is not
commutative, associative, or distributive, achieving the same results
twice can be a matter of luck.  Similar challenges arise when porting
a code from one hardware or software platform to another.

The IEEE Standards for computer arithmetic resulted in significant
improvements in numerical reproducibility on single processors when they were
introduced in the 1970s.  Some work is underway on extending similar
reproducibility to parallel computations, for example in the Intel
Mathematics Kernal Library (MKL), which can use used to provide parallel
reproducibility for mathematical computations.

A number of interesting computational applications have recently been
identified that require even more precision than the 15-digit (64-bit) IEEE
arithmetic that is now standard on modern computer processors.  Such
applications pose even more demanding requirements for justification of the
level of precision used, and for certification of the final results.  One
solution that was suggested at the workshop is to utilize some form of
higher precision arithmetic, such as Kahan’s summation or “double-double”
arithmetic [Bailey2012].  In many cases, such higher precision arithmetic
need only be used in global summations or other particularly sensitive
operations, so that the overall runtime is not greatly expanded.  One
workshop participant reported (Robey2011) that this change has dramatically
increased reproducibility in numerous computational physics applications at
the Los Alamos and Livermore National Laboratories.  
But it is clear that this solution will not work for all applications, and,
in any event, additional study and research is in order.  Certainly the
available tools for high-precision computation need to be significantly
refined so as to be usable and highly reliable for a wide range of users.

These issues must be addressed with deliberate care by authors of
manuscripts.  They must be careful to state in their paper what levels of
numerical reproducibility are appropriate for their task, and then to
provide detailed justification that the level of numeric precision (32-bit,
64-bit or higher precision) is appropriate.

Additional issues in this general arena include: (a) floating-point
standards and whether they being adhered to on the platform in question, (b)
changes that result from different levels of optimization, (c) changes that
result from employing library software, (d) verification of results, and (e)
fundamental limits of numerical reproducibility – what are reasonable
expectations and what are not.

The foundation of numerical reproducibility is also grounded in the
computing hardware and in the software stack. Studies on silent data
corruption (SDC)  have documented SDC in field testing, e.g.
[Constaninescu2000, Kola2005, Panzer-Steindel2007, Michalak2010, Li2010,
Autran2010], testing with proton and neutron beams, e.g. [Michalak2012 and
citations therein], and in other types of studies, e.g.
[Constantinescu2005], with suggestions that SDC may be an increasing issue
with new technologies [Borkar2012].  

Field data on supercomputer DRAM memory failures have shown that advanced
error correcting codes (ECC) are required [Sridharan2012] and technology
roadmaps suggest this problem will only get worse in the coming years.
Designing software that can do some or all of identification, protection,
and correction will become increasingly important.  Still, there is much
work being done to quantify the problem on current and next generation
hardware and approaches to addressing it.  Several United States and
international governmental reports have been produced on the need for,
outlining ongoing research in, and proscribing roadmaps [InterAgency2012,
DarpaResilience2009, TowardsExascaleResilience2009, HECResilience2009].

These foundational components set a limit to the achievable reproducibility
and make us aware that we must continually assess just how reproducible our
methods really are.

\section{Tools to aid in reproducible research} \label{sec:tools}

A principle impediment to the practice of reproducible research has been the
lack of tools that make it easy to work reproducibly.  A number of tools
that address this need have recently emerged.  In this section, we briefly
categorize and describe the tools available to enable replicable or
reproducible research.  This list is not meant to be exhaustive but rather
to give a few examples in each category, and mainly
represents tools that were discussed at the workshop. 
The wiki contains links to these and other tools.

\comment{Also would be useful
to derive useful/essential tool attributes for reproducibility in different
settings}

Dozens of tools were mentioned in workshop presentations, ranging
across different domains and addressing different aspects of
reproducibility.  They can be loosely organized into groups on authoring,
provenance, computational environments, and theorem proving.  

{\bf Literate programming, authoring, and publishing tools.} These tools enable
users to write and publish documents that integrate the text and figures
seen in reports with code and data used to generate both text and graphical
results. In contrast to notebook-based tools discussed below, 
this process is typically not
interactive, and requires a separate compilation step. Tools that enable
literate programming include both programming-language-specific tools such
as WEB, Sweave, and knitr, as well as programming-language-independent tools
such as Dexy, Lepton, and noweb. Other authoring environments include SHARE,
Doxygen, Sphinx, CWEB, and the Collage Authoring Environment.



{\bf Tools that define and execute structured computation and track
provenance.}
Provenance refers to the tracking of chronology and origin of research
objects, such as data, source code, figures, and results. Tools that record
provenance of computations include VisTrails, Kepler, Taverna, Sumatra,
Pegasus, Galaxy, Taverna, Workflow4ever, and Madagascar.

{\bf Integrated tools for version control and collaboration.}  Tools that track
and manage work as it evolves facilitate reproducibility among a group of
collaborators.  With the advent of version control systems (e.g., Git,
Mercurial, SVN, CVS), it has become easier to track the investigation of new
ideas, and collaborative version control sites like Github, Google Code,
BitBucket, and Sourceforge enable such ideas to be more easily shared.
Furthermore, these web-based systems ease tasks like code review and feature
integration, and encourage collaboration.

{\bf Tools that express computations as notebooks.} These tools represent
sequences of commands and calculations as an interactive worksheet with
pretty printing and integrated displays, decoupling content (the data,
calculations) from representation (PDF, HTML, shell console), so that the
same research content can be presented in multiple ways. Examples include
both closed-source tools such as MATLAB (through the publish and app
features), Maple, and Mathematica, as well as open-source tools such as
IPython, Sage, RStudio (with knitr), and TeXmacs.


{\bf Tools that capture and preserve a software environment.}  A major challenge
in reproducing computations is installing the prerequisite software
environment. New tools make it possible to exactly capture the computational
environment and pass it on to someone who wishes to reproduce a computation.
For instance, VirtualBox, VMWare, or Vagrant can be used to construct a
virtual machine image containing the environment.  These images are
typically large binary files, but a small yet complete text description (a
recipe to create the virtual machine) can be stored in their place using
tools like Puppet, Chef, Fabric, or shell scripts. Blueprint analyzes the
configuration of a machine and outputs its text description. ReproZip
captures all the dependencies, files and binaries of the experiment, and
also creates a workflow specification for the VisTrails system in order to make the
execution and exploration process easier. Application virtualization tools,
such as CDE (Code, Data, and Environment), attach themselves to the
computational process in order to find and capture software dependencies.

Computational environments can also be constructed and made available in the
cloud, using Amazon EC2, Wakari, RunMyCode and other tools. VCR, or
Verifiable Computational Research, creates unique identifiers for results
that permits their reproduction in the cloud. 
\comment{code/data repos such as mloss.org , datahub.io, figshare.org?}

Another group are those tools that create an integrated software environment
for research that includes workflow tracking, as well as data access and
version control. Examples include Synapse/clearScience and HUBzero
including nanoHUB.


{\bf Interactive theorem proving systems for verifying mathematics and
computation.}  ``Interactive theorem proving'', a method of formal
verification, uses computational proof assistants to construct formal
axiomatic proofs of mathematical claims. Examples include
coq, Mizar, HOL4, HOL
Light, ProofPowerHOL, Isabelle, ACL2, Nuprl, Veritas, and PVS. Notable theorems
such as the Four Color Theorem have been verified in this way, and Thomas
Hales’ Flyspeck project, using HOL Light and Isabelle, aims to obtain a
formal proof of the Kepler conjecture.
\comment{MathML, MetaMath? and Probabilistically checkable proofs and codes? The
Open Theory Project?}
\todo{Can someone put these more into the context of reprodubility?}

While we have organized these tools into broad categories, it is important to
note that users often require a collection of tools depending on their
domain and the scope of reproducibility desired.  For example, capturing
source code is often enough to document algorithms, but to
replicate results on high-performance computing resources, for example, the
build environment or hardware configuration are also important ingredients.  
Such concerns have been categorized in terms of
the depth, portability, and coverage of reproducibility desired [Freire2012].

The development of software tools enabling reproducible research is a new
and rapidly growing area of research [cite http://stodden.net/AMP2011/ ]. We
think that the difficulty of working reproducibly will be significantly
reduced as these and other tools continue to be adopted and improved.  The
scientific, mathematical, and engineering communities should encourage the
development of such tools by valuing them as significant contributions to
scientific progress.

\section{The teaching and training of reproducibility skills}
\label{sec:teaching2}
The breakout group on Teaching identified the following topics as ones that
instructors might consider including in a course on scientific computing
with an emphasis on reproducibility.  Some subset of these might be
appropriate for inclusion in many other courses.

\begin{itemize} 
\item version control and use of online repositories,
\item modern programming practice including unit testing and regression testing,
\item maintaining “notebooks” or "research compendia",
\item recording the provenance of final results relative to code and/or data,
\item numerical / floating point reproducibility and nondeterminism,
\item reproducibility on parallel systems,
\item dealing with large datasets,
\item dealing with complicated software stacks and use of virtual machines,
\item documentation and literate programming,
\item IP and licensing issues, proper citation and attribution.
\end{itemize} 

The fundamentals/principles of reproducibility can and should taught already
at the undergraduate level. However, care must be taken to not overload the
students with technicalities whose need is not clear from the tasks assigned
to them.  Collaborative projects/assignments can be a good motivation.

\todo{Other possible appendices? For example...
\begin{itemize}
\item Issues related to experimental math and/or theorem proving.
\item Relation to V\&V, UQ, 
\item Also include a snapshot of wiki as another appendix?
\end{itemize} 
}

% Small set of references at end of main document, before Appendices.

\end{document}
